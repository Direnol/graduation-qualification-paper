\chapter{РАБОЧЕЕ ОКРУЖЕНИЕ}


Операционная система \textbf{Ubuntu 18.04} была выбрана в качестве рабочего
окружения по таким техническим параметрам:

\begin{itemize}
    \item Современное программное обеспечение;
    \item Свободное программное обеспечение;
    \item Удобная поставка пакетов;
    \item Совместимость различных библиотек меджу собой;
    \item Стабильность окружения;
    \item Простота настройки системы.
\end{itemize}

Также, для простоты сборки и уверенности в том, что другие разработчики смогут
воспроизвести рабочее окружение с максимальной точностью, и исключить
возможность появления проблем связанных с индивидуальным окружением системы,
используется \texttt{docker}, что также позволит изолировать процесс настройки
окружения и разработки.

В листинге~\ref{docker:image} представлен \texttt{Dockerfile} для сборки образа,
который будет использован в качестве рабочей среды. Для того чтобы собрать
образ~--~требуется выполнить команду~\ref{docker:build}, заранее поместив
содержимое листинга~\ref{docker:image} в файл~\texttt{Dockerfile}.

\begin{lstlisting}[caption={Сборка образа}, label={docker:build}]
$ docker build -t build_env .
\end{lstlisting}

Для запуска контейнера необходимо выполнить команду~\ref{docker:run}. При создании
образа мы указали, что имеется \texttt{volume}~--~ точка монтирования, и она же
является рабочим каталогом. А также, при запуске контейнера, мы использовали
опцию \texttt{-v}, которая позволяет смонтировать в контейнер рабочий каталог и
иметь общие ресурсы: файлы, каталоги.

\begin{lstlisting}[caption={Запуск контейнера}, label={docker:run}]
$ docker run -it -v ${PWD}:/home/user/project build_env bash
\end{lstlisting}

Все последующие команды выполняются в контейнере.
