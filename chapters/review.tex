
\chapter{ОБЗОР ПРЕДМЕТНОЙ ОБЛАСТИ}
\section{Основные понятия}

\subsection{Прокси --- сервер}
Прокси --- сервер (proxy-server) – это особый тип приложения, которое
выступает в роли посредника между клиентскими и серверными частями
распределенных сетевых приложений. Предполагается, что клиенты принадлежат
внутренней (защищаемой) сети, а серверы – внешней (потенциально опасной).

\subsection{HTTP}
\hyperlink{http}{HTTP} --- это прокол передачи данных. В настоящее время существует для передачи
произвольных данных и подразумевает наличие клиента и сервера~\cite{http}.
Клиенты инициализирует запрос, а серверы – ожидают соединения для передачи
запроса, производят какие-либо действия и посылают результат обратно. Прокол
используется во Всемирной паутине для получения информации с веб-сайтов. А
также протокол используется для транспорта других протоколов с прикладным
уровнем. Основным объектом манипуляции является ресурс, на который указывается URI.

\subsection{HTTP сервер}
Сервер распознавания речи речи включает в себя HTTP-сервер, с помощью которого
будет реализован внешний интерфейс приложения (\hyperlink{api}{API}), чтобы другие
подсистемы могли взаимодействовать с ASR.

Протокол HTTP --- это протокол прикладного уровня, который имеет:
\begin{enumerate}
    \item Строка запроса.
    \item Заголовки.
    \item Данные, отделённые от заголовков пустой строкой.
\end{enumerate}

\begin{lstlisting}[caption={Общая структура HTTP протокола}, label={lst:http:struct}]
<Method> <URI> <Version>\r\n
Host: <example.com>\r\n
User-Agent: <agent>\r\n
Header1: <val1>\r\n
Header2: <val2>\r\n
...
HeaderN: <valN>\r\n
\r\n
Payload
\end{lstlisting}

В листинге~\ref{lst:http:struct} представлена общая структура протокола HTTP.
\begin{itemize}
    \item Method --- метод обращения к серверу.
    \begin{itemize}
        \item GET --- получить от сервера данные
        \item POST --- загрузить данные на сервер
        \item PUT --- изменить данные на сервере
        \item DELETE --- удалить данные на сервере
        \item MOVE --- переместить данные
    \end{itemize}
    \item URI --- строка, представляющая путь и аргументы.{\tt URI=URL?QUERY}
    \begin{itemize}
        \item URL --- строка описывающая путь (\texttt{/some/path}).
        \item QUERY --- строка после знака <<\verb!?!>>, что означает наличие аргументов,
            содержащая набор ключ=значение, разделённая знаком \texttt{\&} \\
            <<\verb!key1=val1\&key2=val2\&...\&keyN=valN!>>
    \end{itemize}
    \item Versoin --- версия протокола(\texttt{HTTP/1.1} или \texttt{HTTP/1.0}).
    \item Header --- имя(ключ) заголовка.
    \begin{itemize}
        \item Host --- заголовок, указывающий имя сервера, к которому происходит обращение.
        \item User-Agent --- заголовок, указывающий имя клиента.
    \end{itemize}
    \item val --- значения соотвествующего заголовка.
    \item Payload --- данные передаваемые от клиента к серверу.
\end{itemize}

Также существует специальный заголовок \texttt{Transfer-Encoding}, который при
значении \texttt{chunked} позволяет сообщить, что будет происходить потоковая передача.
При такой передачи данных клиент посылает размер в шестнадцатеричной системе
счисления следующего куска данных, после чего отсылает данный кусок данных, и
повторяет данную процедуру необходимое число раз. \mbox{Когда} клиент закончил передачу данных,
он сообщает серверу, что следующий кусок данных содержит 0 байт.

\begin{lstlisting}[caption={Пример \texttt{Transfer-Encoding: chunked}}, label={lst:http:chunked}]
POST /path/to/file HTTP/1.1\r\n
Host: example.com\r\n
User-Agent: client\r\n
Transfer-Encoding: chunked\r\n
\r\n
a\r\n
<10 байт>\r\n
f\r\n
<15 байт>\r\n
0\r\n
\end{lstlisting}

В ином случае, когда не указан \texttt{Transfer-Encoding}, клиент должен \mbox{сообщить}
размер посылаемых данных в байтах с помощью заголовка \texttt{Content-Type}.

\subsection{Форматы аудио файлов}
Существует множество различных аудио форматов. Наиболее часто используются такие
форматы как MP3 (MPEG-2 Audio Layer III) и WAV. Тип формата обычно определяется
расширением файла (то, что идет после точки в имени файла .mp3, .wav, .ogg, .wma)~\cite{audio}.

Кодек – это определенный алгоритм кодирования и сжатия данных в аудио-формат.
Для некоторых типов файлов кодек однозначно определен. Например в формате mp3
всегда используется кодек MPEG Layer-3, а в формате mp4 могут быть использованы
разные кодеки~\cite{audio}.

Ниже представлен список используемых форматов:
\begin{itemize}
    \item \hyperlink{wav}{WAV} --– один из первых аудио-форматов. Обычно используется для хранения
    несжатых аудио-записей (PCM), идентичных по качеству звука записям на
    компакт-дисках (audio-CD). В среднем одна минута звука в формате wav занимает
    около 10 мегабайт~\cite{audio}.
    \item \hyperlink{pcm}{PCM} --- цифровое представление аналогового сигнала, в
    котором сигнал сначала дискретизируется (то есть, величина сигнала фиксируется
    в определённые интервалы времени) затем квантуется (то есть, величина сигнала
    в каждый из дискретных интервалов представляется наиболее близким значением из
    допустимых), а затем кодируется с помощью цифрового представления~\cite{pcm}.
\end{itemize}

\subsection{Linux}
\subsection{Systemd}

\section{Инструменты для разработки}

\subsection{C++}
С++ --- язык общего назначения и задуман для того, чтобы настоящие программисты получили
удовольствие от самого процесса программирования. За исключением второстепенных деталей он
содержит язык С как подмножество. Язык С расширяется введением гибких и эффективных средств,
предназначенных для построения новых типов~\cite{cpp}


\subsection{Nginx}
Nginx [engine x] —-- это HTTP-сервер и обратный прокси-сервер, почтовый
прокси-сервер, а также TCP/UDP прокси-сервер общего назначения, изначально
написанный Игорем Сысоевым~\cite{nginx}.

\subsection{libevent}
libevent --- библиотека для асинхронного уведомления о событиях со всроенным
HTTP~--~сервером.

\subsection{CMake}

CMake — кроcсплатформенная утилита для автоматической сборки программы из
исходного кода. При этом сама CMake непосредственно сборкой не занимается,
а представляет из себя front-end. В качестве back-end`a могут выступать
различные версии make и Ninja~\cite{cmake}.

\section{Иснтрументы для распознавания речи}

\subsection{Kaldi}
Kaldi --- это набор инструментов для распознавания речи, свободно доступный по
лицензии Apache. Kaldi стремится предоставить программное обеспечение, которое
является гибким и расширяемым.~\cite{kaldi}.

Kaldi предоставляет современный и гибкий код, написанный на \texttt{C++}, который
легко модифицировать и расширять. Важные особенности включают в себя:

\begin{itemize}
    \item Интеграцию на уровне кода с библиотекой для конструирования, комбинирования
        и поиска взвешенных конечных преобразователей (libfst).
    \item Обширную поддержку линейной алгебры.
    \item Матричную библиотеку, которая упаковывает стандартные процедуры BLAS и LAPACK.
    \item Расширяемый дизайн.
    \item Алгоритмы представляют общий интерфейс, насколько это возможно.
    \item Универсальные алгоритмы.
\end{itemize}

Также Kaldi предоставляет готовые рецепты для построения систем \\ \mbox{распознавания} речи,
которые работают из широко доступных баз данных.

Он поддерживает линейное преобразование, MMI,
усиленное MMI и MCE дискриминационное обучение, дискриминационное обучение в
пространстве признаков и глубокие нейронные сети.

\subsection{CMU Sphinx}
CMU Sphinx --- является общим термином для описания группы систем распознавания
речи, разработанных в Университете Карнеги-Меллона. Они включают в себя серию
распознавателей речи и акустического модельного тренажера~\cite{sphinx}.

CMUSphinx собирает более 20 лет исследований CMU. Все преимущества сложно перечислить,
но лишь некоторые из них:
\begin{itemize}
    \item Современные алгоритмы распознавания речи для эффективного распознавания речи.
    \item Инструменты CMUSphinx разработаны специально для низкоресурсных платформ.
    \item Гибкая конструкция.
    \item Фокус на разработке практических приложений, а не на исследованиях.
    \item Поддержка многих языков.
    \item Активное сообщество.
    \item Активная разработка и график выпуска.
    \item Широкий спектр инструментов для многих целей, связанных с распознаванием речи:
    \begin{itemize}
        \item определение ключевых слов
        \item выравнивание
        \item оценка произношения
    \end{itemize}
\end{itemize}
