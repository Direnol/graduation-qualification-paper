
\chapter{ОБЗОР ПРЕДМЕТНОЙ ОБЛАСТИ}

\section{Теоретические сведения}

\subsection{Прокси-сервер}
Прокси-сервер (proxy-server) – это особый тип приложения, которое
выступает в роли посредника между клиентскими и серверными частями
распределенных сетевых приложений. Предполагается, что клиенты принадлежат
внутренней (защищаемой) сети, а серверы – внешней (потенциально опасной).

\subsection{Протокол передачи данных HTTP}
\hyperlink{http}{HTTP} --- это прокол передачи данных. В настоящее время существует для передачи
произвольных данных и подразумевает наличие клиента и сервера~\cite{http}.
Клиенты инициализирует запрос, а серверы – ожидают соединения для передачи
запроса, производят какие-либо действия и посылают результат обратно. Прокол
используется во Всемирной паутине для получения информации с веб-сайтов. А
также протокол используется для транспорта других протоколов с прикладным
уровнем. Основным объектом манипуляции является ресурс, на который указывается URI.

\subsection{HTTP-сервер}
Сервер распознавания речи речи включает в себя HTTP-сервер, с помощью которого
будет реализован внешний интерфейс приложения (\hyperlink{api}{API}), чтобы другие
подсистемы могли взаимодействовать с ASR.

Протокол HTTP --- это протокол прикладного уровня, который имеет:
\begin{enumerate}
    \item Строка запроса.
    \item Заголовки.
    \item Данные, отделённые от заголовков пустой строкой.
\end{enumerate}

В листинге~\ref{lst:http:struct} представлена общая структура протокола HTTP.
\begin{itemize}
    \item Method --- метод обращения к серверу.
    \begin{itemize}
        \item GET --- получить от сервера данные
        \item POST --- загрузить данные на сервер
        \item PUT --- изменить данные на сервере
        \item DELETE --- удалить данные на сервере
        \item MOVE --- переместить данные
    \end{itemize}
    \item URI --- строка, представляющая путь и аргументы.{\tt URI=URL?QUERY}
    \begin{itemize}
        \item URL --- строка описывающая путь (\texttt{/some/path}).
        \item QUERY --- строка после знака <<\verb!?!>>, что означает наличие аргументов,
            содержащая набор ключ=значение, разделённая знаком \texttt{\&} \\
            <<\verb!key1=val1\&key2=val2\&...\&keyN=valN!>>
    \end{itemize}
    \item Versoin --- версия протокола(\texttt{HTTP/1.1} или \texttt{HTTP/1.0}).
    \item Header --- имя(ключ) заголовка.
    \begin{itemize}
        \item Host --- заголовок, указывающий имя сервера, к которому происходит обращение.
        \item User-Agent --- заголовок, указывающий имя клиента.
    \end{itemize}
    \item val --- значения соотвествующего заголовка.
    \item Payload --- данные передаваемые от клиента к серверу.
\end{itemize}

\begin{lstlisting}[caption={Общая структура HTTP протокола}, label={lst:http:struct}]
<Method> <URI> <Version>\r\n
Host: <example.com>\r\n
User-Agent: <agent>\r\n
Header1: <val1>\r\n
Header2: <val2>\r\n
...
HeaderN: <valN>\r\n
\r\n
Payload
\end{lstlisting}
Также существует специальный заголовок \texttt{Transfer-Encoding}, который при
значении \texttt{chunked} позволяет сообщить, что будет происходить потоковая передача.
При такой передачи данных клиент посылает размер в шестнадцатеричной системе
счисления следующего куска данных, после чего отсылает данный кусок данных, и
повторяет данную процедуру необходимое число раз. \mbox{Когда} клиент закончил передачу данных,
он сообщает серверу, что следующий кусок данных содержит 0 байт. Пример использования
данного заголовка представлен \linebreak в~листинге~\ref{lst:http:chunked}.

\begin{lstlisting}[caption={Пример \texttt{Transfer-Encoding: chunked}}, label={lst:http:chunked}]
POST /path/to/file HTTP/1.1\r\n
Host: example.com\r\n
User-Agent: client\r\n
Transfer-Encoding: chunked\r\n
\r\n
a\r\n
<10 байт>\r\n
f\r\n
<15 байт>\r\n
0\r\n
\end{lstlisting}

В ином случае, когда не указан \texttt{Transfer-Encoding}, клиент должен \mbox{сообщить}
размер посылаемых данных в байтах с помощью заголовка \texttt{Content-Type}.

\subsection{Форматы аудио файлов}
Существует множество различных аудио форматов. Наиболее часто используются такие
форматы как MP3 (MPEG-2 Audio Layer III) и WAV. Тип формата обычно определяется
расширением файла (то, что идет после точки в имени файла .mp3, .wav, .ogg, .wma)~\cite{audio}.

Кодек – это определенный алгоритм кодирования и сжатия данных в аудио-формат.
Для некоторых типов файлов кодек однозначно определен. Например в формате mp3
всегда используется кодек MPEG Layer-3, а в формате mp4 могут быть использованы
разные кодеки~\cite{audio}.

Ниже представлен список используемых форматов:
\begin{itemize}
    \item \hyperlink{wav}{WAV} --– один из первых аудио-форматов. Обычно используется для хранения
    несжатых аудио-записей (PCM), идентичных по качеству звука записям на
    компакт-дисках (audio-CD). В среднем одна минута звука в формате wav занимает
    около 10 мегабайт~\cite{audio}.
    \item \hyperlink{pcm}{PCM} --- цифровое представление аналогового сигнала, в
    котором сигнал сначала дискретизируется (то есть, величина сигнала фиксируется
    в определённые интервалы времени) затем квантуется (то есть, величина сигнала
    в каждый из дискретных интервалов представляется наиболее близким значением из
    допустимых), а затем кодируется с помощью цифрового представления~\cite{pcm}.
\end{itemize}

\subsection{Операционная система Linux}
Linux –-- операционные системы, которые основаны на свободном и открытом
программном обеспечении. Операционные системы на базе ядра Linux представляют
собой группу Unix-подобных операционных систем~\cite{linux}.

ОС Linux может быть установлена на большое количество компьютерных устройств:
\begin{itemize}
    \item мобильные телефоны
    \item планшетные компьютеры
    \item маршрутизаторы
    \item игровые консоли
    \item настольные компьютеры
    \item мэйнфреймы
    \item суперкомпьютеры
\end{itemize}

У Linux нет единой «официальной» комплектации, так как она поставляется в виде
дистрибутивов. Дистрибутивы Linux создаются на основе одноименного ядра,
библиотек и системных программ, которые разрабатываются в рамках проекта GNU.
ОС Linux является одним из наиболее ярких примеров свободного и открытого
программного обеспечения, так как исходный код операционной системы находится в
свободном доступе и может быть использован в других проектах, если они также как
и ОС выпускаются под лицензией GNU General Public License.

\subsection{Менеджер системы Systemd}
Systemd представляет собой менеджер системы и сервисов для Linux, совместимый с
SysV и \hyperlink{lsb}{LSB} сценариями. systemd обеспечивает возможности агрессивного
распараллеливания, использует сокет и D-Bus для начала активации сервиса по
требованию, запуск демонов в фоновом режиме, отслеживает процессы, использующие
Linux cgroups, поддерживает мгновенные снимки и восстановление состояния системы,
поддерживает подключения и автоматическое монтирование точек и реализует разработку
логики транзакционных зависимостей службы управления. Он может работать как
замена для Sysvinit~\cite{systemd}.

\subsection{Технология JSGF}
Технология JSpeech Grammar Format (JSGF)~---~платформонезависимая текстовое
представление грамматик для использования в компьютерном распознавании речи.
JSGF перенимает стиль и соглашения языка Java в дополнение к использованию
традиционной грамматической нотации~\cite{jsgf}.

Грамматика~---~это формальные правила, которые описывают простые правила
построения предложений. Лексемы, полученные на предыдущем шаге пытаются
сопоставиться с грамматикой и если удачно, то выводится результат.

\section{Инструменты для разработки}

\subsection{Язык программирования C++}
С++ --- компилируемый, статически типизированный язык программирования общего назначения.
Поддерживает такие парадигмы программирования, как процедурное программирование,
объектно-ориентированное программирование, обобщённое программирование.
Язык имеет богатую стандартную библиотеку, которая включает в себя
распространённые контейнеры и алгоритмы, ввод-вывод, регулярные выражения,
поддержку многопоточности и другие возможности. C++ сочетает свойства как
высокоуровневых, так и низкоуровневых языков. В сравнении с его предшественником~—-~
языком C,~-—~наибольшее внимание уделено поддержке объектно-ориентированного и
обобщённого программирования~\cite{cpp}.


\subsection{Прокси-сервер Nginx}
Nginx [engine x] —-- это HTTP-сервер и обратный прокси-сервер, почтовый
прокси-сервер, а также TCP/UDP прокси-сервер общего назначения, изначально
написанный Игорем Сысоевым~\cite{nginx}.

\subsection{Библиотека libevent}
libevent --- библиотека для асинхронного уведомления о событиях со всроенным
HTTP~--~сервером.

\subsection{Утилита для сборки CMake}

CMake — кроcсплатформенная утилита для автоматической сборки программы из
исходного кода. При этом сама CMake непосредственно сборкой не занимается,
а представляет из себя front-end. В качестве back-end`a могут выступать
различные версии make и Ninja~\cite{cmake}.

\subsection{Система контроля версий Git}
GIT – Git – это разновидность VCS (Version Control System).
А VCS~–~это программа для работы с постоянно изменяющейся информацией. Такое ПО
может хранить множество версий одного и того же файла (документа) и возвращаться
к более раннему состоянию (версии)~\cite{git}.

\subsection{Программная платформа Docker}
Docker – это программная платформа для быстрой сборки, отладки и развертывания
приложений. Docker упаковывает ПО в стандартизованные блоки, которые называются
контейнерами. Каждый контейнер включает все необходимое для работы приложения:
библиотеки, системные инструменты, код и среду исполнения. Благодаря Docker
пользователи могут быстро развертывать и масштабировать свои приложения в любой
среде и сохранять уверенность в том, что код будет работать~\cite{docker}.

\subsection{Библиотека ATLAS}
\hyperlink{atlas}{ATLAS} — программная библиотека для линейной алгебры. Она
представляет собой реализацию BLAS для языков Си и Фортран~\cite{atlas}.

Рекомендуется как высокопроизводительная реализация BLAS для различных платформ.

\subsection{Программа Bison}
GNU Bison — программа, предназначенная для автоматического создания
синтаксических анализаторов по данному описанию грамматики. Программа bison
относится к свободному ПО, разработана в рамках проекта GNU и портирована под
все традиционные операционные системы. Программа bison во многом совместима с
подобной программой yacc. Обычно используется в комплексе с лексическим
анализатором flex~\cite{bison}.

\subsection{Утилита для сборки Autoconf}
Autoconf — утилита, использующая шаблоны и наборы макросов на языке m4, для
создания конфигурационных скриптов (configure), которые автоматически настраивают
пакеты с исходным кодом для работы в Unix-подобных операционных системах~\cite{autoconf}.

\subsection{Инструмент Swig}
SWIG (англ. simplified wrapper and interface generator) — свободный инструмент
для связывания программ и библиотек, написанных на языках C и C++, с
интерпретируемыми или компилируемыми языками.

Основная цель: обеспечение возможности вызова
функций, написанных на одних языках, из кода на других языках. Программист
создаёт файл \texttt{.i} с описанием экспортируемых функций; SWIG генерирует исходный код
для склеивания C/C++ и нужного языка, создаёт исполняемый файл~\cite{swig}.

\subsection{Служебная программа curl}
Сurl используется в командной строке или скриптах для передачи данных с использованием
многих протоколов: HTTP, HTTPS, FTP, FTPS, SCP, LDAP и др.

\section{Иснтрументы для распознавания речи}

\subsection{Набор библиотек Kaldi}
Kaldi --- это набор инструментов для распознавания речи, свободно доступный по
лицензии Apache. Kaldi стремится предоставить программное обеспечение, которое
является гибким и расширяемым.~\cite{kaldi}.

Kaldi предоставляет современный и гибкий код, написанный на \texttt{C++}, который
легко модифицировать и расширять. Важные особенности включают в себя:

\begin{itemize}
    \item Интеграцию на уровне кода с библиотекой для конструирования, комбинирования
        и поиска взвешенных конечных преобразователей (libfst).
    \item Обширную поддержку линейной алгебры.
    \item Матричную библиотеку, которая упаковывает стандартные процедуры BLAS и LAPACK.
    \item Расширяемый дизайн.
    \item Алгоритмы представляют общий интерфейс, насколько это возможно.
    \item Универсальные алгоритмы.
\end{itemize}

Также Kaldi предоставляет готовые рецепты для построения систем \\
\mbox{распознавания} речи, которые работают из широко доступных баз данных.

Он поддерживает линейное преобразование, MMI,
усиленное MMI и MCE дискриминационное обучение, дискриминационное обучение в
пространстве признаков и глубокие нейронные сети.

\subsection{Набор библиотек CMU Sphinx}
CMU Sphinx --- является общим термином для описания группы систем распознавания
речи, разработанных в Университете Карнеги-Меллона. Они включают в себя серию
распознавателей речи и акустического модельного тренажера~\cite{sphinx}.

CMUSphinx собирает более 20 лет исследований CMU. Все преимущества сложно перечислить,
но лишь некоторые из них:
\begin{itemize}
    \item Современные алгоритмы распознавания речи для эффективного распознавания речи.
    \item Инструменты CMUSphinx разработаны специально для низкоресурсных платформ.
    \item Гибкая конструкция.
    \item Фокус на разработке практических приложений, а не на исследованиях.
    \item Поддержка многих языков.
    \item Активное сообщество.
    \item Активная разработка и график выпуска.
    \item Широкий спектр инструментов для многих целей, связанных с распознаванием речи:
    \begin{itemize}
        \item определение ключевых слов
        \item выравнивание
        \item оценка произношения
    \end{itemize}
\end{itemize}
