\chapter{ПРОЕКТИРОВАНИЕ СИСТЕМЫ}

\section{Структура взаимодействия}
\pic[h!][0.7]{structure.png}{asr:struct}{Структура взаимодействия подсистем}

На рисунке~\ref{fig:asr:struct} изображена общая структура взаимодействия подсистем.
Рассмотрим её подробнее:
\begin{itemize}
    \item Подсистема ASR состоит из двух компонент;
    \begin{enumerate}
        \item HTTP-сервер~---~ отвечает за приём http-запросов и обработку api;
        \item Kaldi~---~реализует процесс распознавания речи в текст.
    \end{enumerate}
    \item Nginx~---~выполняет роль прокси-сервера для балансировки запросов и
        нормирование протокола HTTP;
    \item MSR~---~является промежуточным звеном между IVR и ASR. Осуществялет маршрутизацию
        аудио-потока, а также формирует HTTP-запрос с корректными заголовками;
    \item IVR~---~подсистема голосового меню при взаимодействии с ядром шлюза ECSS10
        осуществялется маршрутизация звонка;
    \item ECSS10~---~шлюз ECSS10 осуществялет маршрутизацию и контроль над звонками;
    \item Пользователь~---~клиент позвонивший на данный шлюз, перенаправленный на
        общение с подсистемой IVR;
\end{itemize}

\section{Параметры запуска}
\begin{lstlisting}[caption={Обработка аргументов},label={asr:init:options},language=C++]
ostringstream usage;
usage << "Description...";
auto _usage = usage.str();
auto po = make_unique<ParseOptions>(_usage.c_str());

\end{lstlisting}

Для того, чтобы управлять поведение подсистемы~---~воспользуемся встроенными
средствами \textit{Kaldi} для чтения файлов конфигурации и параметров командной
строки. Для этого необходимо создать объект типа \texttt{ParseOptions}, при этом
необходимо передать строковый аргумент, который описывает как использовать данную
подсистему, что изображено в листинге~\ref{asr:init:options}.

\begin{lstlisting}[caption={Пример объявление параметров},label={asr:init:parameters},language=C++]
this->chunk_length_secs = 0.18;
this->output_period = 1;
this->address = "0.0.0.0";
this->port = 9000;
this->thread_count = 1;
po->Register("chunk-length", &chunk_length_secs,
                "Length of chunk size in seconds, that we process.");

po->Register("output-period", &output_period,
                "How often in seconds, do we check for changes in output.");

po->Register("num-threads-startup", &g_num_threads,
                "Number of threads used when initializing iVector extractor.");

po->Register("nnet3-in", &nnet3_rxfilename, "nnet3-in");
po->Register("fst-in", &fst_rxfilename, "fst-in");
po->Register("word-symbol-table", &word_syms_filename, "word-symbol-table");

this->feature_opts.Register(po.get());
this->decodable_opts.Register(po.get());
this->decoder_opts.Register(po.get());
this->endpoint_opts.Register(po.get());

\end{lstlisting}

Далее необходимо объявить параметры, пример показан в листинге~\ref{asr:init:parameters}.
Для того, чтобы добавить новый параметр, требуется создать переменную и присвоить ей
значение по умолчанию. После чего зарегестрировать данную параметр, передав:
имя параметра, адрес переменной и описание. После объявление пользовательских
параметров, требуется добавить параметры для \textit{Kaldi}~---~последние 4 строки в
листинге~\ref{asr:init:parameters}.

По итогу были добавлены следующие параметры:
\begin{enumerate}
    \item \texttt{address}~---~Адрес для прослушивания;
    \item \texttt{port}~---~Порт для принятия соединений;
    \item \texttt{workers}~---~Количество рабочих потоков;
    \item \texttt{backlog}~---~Размер очереди при установлении tcp-соединения;
\end{enumerate}
Также, \textit{Kaldi} регистрирует параметры для установки частоты дискретизации,
путь до языковой модели и длину чанка в секунду.

Большая часть переменных содержится в классе, как приватные члены, для того, чтобы
уменьшить количество памяти при инициализации \textit{Kaldi}.
Данный процесс осуществляется функцией \texttt{kaldi\_init(int argc, char **argv)}.
\section{HTTP-сервер}

Для реализации HTTP-сервера используется библиотека \textit{libevent}, которая
имеет дополнительный интерфейс для создания HTTP-сервера.

Данная библиотека имеет возможность организовать многопоточное обслуживание клиентов,
и инициализировать потоки, что позволяет избежать накладных расходов на инициализацию
\textit{Kaldi} при каждом запросе на распознавание, а осуществить её только
при запуске подсистемы.

Рассмторим порядок запуска подсистемы:
\begin{enumerate}
    \item Загрзука файла с параметрами запуска;\label{asr:init:conf}
    \item Создать объект для регистрации событий \texttt{event};\label{libevent:event:create}
    \item Создать http-стркутуру для регистрации событий \texttt{evhttp}
        на основе \texttt{event};\label{libevent:http:create}
    \item Зарегестрировать сетевой адрес и порт для прослушивания;\label{libevent:http:bind}
    \item Создать необходимое количество потоков;
    \item Для каждого потока проделать пункт~\ref{libevent:http:create};
    \item Каждый поток выполняет инициализацию \textit{Kaldi};
    \item Ожидает поступления запросов.
\end{enumerate}

\begin{lstlisting}[caption={Создание базовых объектов для инициализации подсистемы},label={asr:http:init},language=C++]
kaldi_init(argc, argv);

this->base = new ptr_base(event_base_new(), &event_base_free);
if (!base)
    ASR_ERR << "Failed to create new base_event.";

this->http = new ptr_http(evhttp_new(base->get()), &evhttp_free);

if (!http)
    ASR_ERR << "Failed to create new http event";

auto BoundSock = evhttp_bind_socket_with_handle(http->get(), address.c_str(), port);
if (!BoundSock)
    ASR_ERR << "Failed to bind server socket.";

if ((socket = evhttp_bound_socket_get_fd(BoundSock)) == -1)
    ASR_ERR << "Failed to get server socket for next instance.";

this->thread_count--;

\end{lstlisting}

В листинге~\ref{asr:http:init} представлен код конструктора класса, который
выполняет действия, описанные в пунктах~\ref{asr:init:conf}~--~\ref{libevent:http:bind}.

\section{Внешний интерфейс}

Внешний интерфейс приложения осуществляется с помощью http-заголовков. Ниже
првиведён список заголовков, которые определяют подведение подсистемы:
\begin{itemize}
    \item \texttt{Content-Type}~---~тип аудио-потока;
    \begin{itemize}
        \item\texttt{audio/x-pcm}~---~ для \textit{pcm} формата;
        \item\texttt{audio/x-wav}~---~ для \textit{wav} формата.
    \end{itemize}
    \item \texttt{SR-Key}~---~перечисленные через запятую фразы, которые необходимо распознать;
    \begin{itemize}
        \item Если данный заголовок пустой, то вовзращаем всё что было распознано.
    \end{itemize}
\end{itemize}

\begin{lstlisting}[caption={Чтение заголовков},label={asr:http:headers},language=C++]
auto headers_in = evhttp_request_get_input_headers(req);
auto keys = evhttp_find_header(headers_in, "sr-key");
auto type = evhttp_find_header(headers_in, "content-type");

\end{lstlisting}

В листинге~\ref{asr:http:headers} приводится пример чтения http-заголовков. На их
основе решается какой алгоритм действий необходимо выполнить.

\section{Настройка прокси-сервера}

Для настройки прокси-сервера используется \textit{Nginx}, который осуществляет,
при необходимости, балансировку запросов между несколькими экземлярами ASR, к
тому же позволяет обращаться к нему по любой версии протокола HTTP.

\begin{lstlisting}[caption={Пример конфигурации Nginx},label={nginx:conf}]
location ~^\/(speech-recognition|asr) {
    proxy_http_version 1.1;
    proxy_request_buffering off;
    proxy_pass http://localhost:9000;
    proxy_method POST;
}

\end{lstlisting}

В листинге~\ref{nginx:conf} приведена часть конфигурации \textit{Nginx},которая
отвечает за проксирование запросов.

\section{Интеграция Kaldi с HTTP-сервером}
Ранее приводилась небольшая часть кода в листинге~\ref{asr:init:parameters}, где
происходит определение параметров при помощи встроенных средств \textit{Kaldi}.

Разберём подробнее процесс загрузки \textit{Kaldi}, а также метод интеграции
данного инструмента с библиотекой \textit{libevent}.

\begin{lstlisting}[caption={Создание обработчика событий},label=asr:libevent:callback,language=C++]
evhttp_set_gencb(EvHttp.get(), callback_recognize, &sa);
ASR_LOG << "Handler has been loaded...";
event_base_dispatch(EventBase.get());

\end{lstlisting}

Библиотека \textit{libevent}, как ранее было описано, производит инициализацию
необходимых объектов для регистрации и обработки событий, куда регистрируются
http события. Каждому событию определяется обработчик. Так, например, определяется
общий обработчик на все http события~(см. листинг~\ref{asr:libevent:callback}),
после чего запускается бесконечный цикл для ожидания событий.

Стоит отметить, что при реализации используется класс и следующие примеры вызываются
в методах, у которых есть доступ к приватным членам класса. Также используются
<<Resource Acquisition is Initialization>(\hyperlink{raii}{RAII})~---~
захват ресурса есть инициализация, который позволяет создать объект и указать
способ освобождения памяти при уничтожении объекта, когда он выходит за область
видимости.

\begin{lstlisting}[caption={Описание некоторых типов данных}, label={lst:class:types}]
using ptr_base = unique_ptr<event_base, decltype(&event_base_free)>;
using ptr_http = unique_ptr<evhttp, decltype(&evhttp_free)>;
\end{lstlisting}

В листинге~\ref{lst:class:types} представлены типы данных, уникальных указателей,
которые реализуют RAII.

\begin{lstlisting}[caption={структуры для хранения данных},label={asr:server_args},language=C++]
struct kaldi_args {
    OnlineNnet2FeaturePipelineInfo *feature_info{};
    LatticeFasterDecoderConfig *decoder_opts{};
    TransitionModel *trans_model{};
    nnet3::DecodableNnetSimpleLoopedInfo *decodable_info{};
    Fst<StdArc> *fst{};
    BaseFloat chunk_length_secs{};
    BaseFloat output_period{};
    BaseFloat samp_freq{};
    SymbolTable *word_syms{};
    NnetSimpleLoopedComputationOptions *decodable_opts{};
    OnlineEndpointConfig *endpoint_opts{};

};

struct server_args {
//    http
    string *sname{};
    string *save_dir{};
    bool save_rec{};

//    kaldi
    kaldi_args ka_arg{};
};

\end{lstlisting}

На этапе чтения конфигурации все необходимые данные для распознавания речи
были проинициализированы и хранятся в приватных членах класса. Поэтому при
создании обработчика ему передаётся указатель на структуру, которая содержит
указатели на необходимые данные~(см.~листинг~\ref{asr:server_args}). Это позволяет
уменьшить объём памяти для хранения данных.


\section{Конфигурация подсистемы}

Для запуска подсистемы необходимо указать параметры, которые определяют:
\begin{enumerate}
    \item Языковую модель;
    \item Адрес для установления tcp-соединения;
    \item Качество распознавания.
\end{enumerate}

Список параметров првиведён на странице~\pageref{kaldi:options}.
\begin{lstlisting}[caption={Запуск ASR},label={asr:run},language=bash]
$ ecss-asr --config=config --config=exp/tdnn/conf/online.conf

\end{lstlisting}

\begin{lstlisting}[caption={Содежимое config},label={asr:config},language=bash]
# kaldi
--frames-per-chunk=51
--extra-left-context-initial=0
--frame-subsampling-factor=3
--min-active=200
--max-active=10000
--beam=15.0
--lattice-beam=6.0
--acoustic-scale=1.0
# http
--address=0.0.0.0
--port=9000
--sname=ECSS-ASR
--workers=4
# model
--nnet3-in=exp/tdnn/final.mdl
--fst-in=exp/tdnn/graph/HCLG.fst
--word-symbol-table=exp/tdnn/graph/words.txt

\end{lstlisting}


Рассмотрим пример запуска \textit{ASR}, приведённый в листинге~\ref{asr:run}.
Первый параметр определяет путь до конфигурационного файла, где перечислены
остальные опции, это сделано для удобства настройки подсистемы. Следующий
файл конфигурации определяет параметры для языковой модели.

В конфигурационном файле~\ref{asr:config} перечислены опции для запуска по умолчанию.
Параметры, относящиеся к конфигурации \textit{Kaldi}, были получены в ходе
экспулатации подсистемы и на данный момент являются оптимальными для работы с
аудио-потоком постпующим с телефонных апаратов. Другие параметры относятся к
настройке HTTP-сервера и указанию языковой модели.


\section{Создание systemd сервиса}
\begin{lstlisting}[caption={Содежимое ecss-asr.service},label=asr:systemd]
[Unit]
Description=Daemon ecss-asr
Wants=network-online.target ecss-restfs-core.service

[Service]
User=ssw
WorkingDirectory=/usr/lib/ecss/ecss-asr/model/ru-RU
Restart=on-failure
ExecStart=/usr/bin/ecss-asr --config=/usr/lib/ecss/ecss-asr/config \
                            --config=exp/tdnn/conf/online.conf

[Install]
Alias=ecss-asr.service
WantedBy=multi-user.target

\end{lstlisting}

Для того, чтобы подсистема \textit{ASR} работала в режиме демона и была всегда
активна, используется подсистема инициализации \textit{Systemd}, которая
позволяет помимо запуска и перезапуска приложения, также регулировать
привилегия и доступ к ресурсам ОС. В приведённом листинге~\ref{asr:systemd}
показан запуск \textit{ASR} с помощью \textit{Systemd}.

\section{Взаимодействие подсистем между собой}
В ходе общения системы \textit{IVR} и клиента, система задаёт вопросы и сообщает фразы,
которые ожидает услышать от клиента. Для того чтобы сообщить системе \textit{ASR}
какие фразы являются ключевыми, система \textit{MSR} передаёт их в заголовке \texttt{SR-Key},
например \texttt{SR-Key: конечно, да, может быть, нет}.

После чего система \textit{MSR} перенаправляет аудио-поток клиента на \textit{ASR},
при этом заполняя заголовок \texttt{Content-Type} соотвествующим \linebreak
значением: \texttt{audio/x-wav} или \texttt{audio/x-pcm}, в зависимости от
настроек \linebreak системы.

Далее происходит процесс распознавания речи и поиска ключевых слов. При обнаружении
ключевого слово система \textit{ASR} возвращает ответ со статусом 200, в тело которого
вложено распознанное ключевое слово, иначе возвращается ответ со статусом 404.

\section{Упаковка приложения в debian пакет}

Для упрощения поставки данного приложения используется пакетная система \textit{debian}.
Данная технология упрощает процесс развётывания приложения, а также его настройку.

Приложение упаковывается в \textit{.tar.gz} архив, создавая необходимую иерархию
относительно корня системы. Дополнительно имеется архив с установочными скриптами,
которые осуществляют настройку пакета.При установке архив распаковывается
в корень системы, после чего запускаются установочные скрипты, которые
опрашивают пользователя о параметрах запуска приложения, устанавливают в систему
systemd сервисы. Архив с приложением содержит:
\begin{itemize}
    \item Конфигурационный файл;
    \item Исполняемый файл;
    \item Набор необходимых библиотек;
    \item Русскую языковую модель;
    \item Файл systemd сервиса.
\end{itemize}
