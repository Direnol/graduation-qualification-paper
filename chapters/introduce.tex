\chapter{ВВЕДЕНИЕ}

В современном мире всё больше людей пользуются услугами. Как правило,
человеку, чтобы <<заказать>> услугу, требуется позвонить в компанию, которая
предоставляет необходимую услугу. Позвонившему нужно дождаться, когда на его
звонок ответит оператор и проконсультирует его.

С другой стороны -- компания, которая предоставляет услуги, необходимо нанимать
достаточное число операторов в контакт-центр, чтобы увеличить количество
обслуживаемых клиентов, что в свою очередь требует дополнительных затрат.

Для решения данной проблемы было введено интерактивное голосовое меню, которое
позволило снизить затраты человеко-часов на обслуживание звонков,
обрабатывать звонки в нерабочее время, улучшить качество обслуживание за счёт того,
что клиент мог сам выбрать с кем его соединить, уменьшить время ожидания ответа,
получать дополнительную информацию, балансировать звонки между свободными сотрудниками.

На текущий момент основной способ <<общения>> клиента с системой IVR посредством
тонального набора, то есть нажатием цифры на телефоне.

Тональный набор -- набор телефонного номера, при котором каждая цифра кодируется
парой из восьми тональных частот звукового диапазона. Во всех телефонах с
тональным набором обычно используется стандартная двенадцатикнопочная клавиатура.
Каждой строке в матрице клавиатуры соответствует низкочастотный тон, а каждому
столбцу - высокочастотный. При нажатии кнопки два тона суммируются и формируется
двухчастотный сигнал. Все частоты подобраны так, что ни одна из них не является
гармоникой другой. Длительность одного символа при тональном наборе - 40 мс.~\cite{dtmf}
