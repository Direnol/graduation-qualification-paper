\chap{ВВЕДЕНИЕ}

В современном мире всё больше людей пользуются услугами телекоммуникации, которые
позволяют:
\begin{itemize}
    \item Связаться с другим человеком.
    \item Заказать продукт или услугу.
    \item Узнать информацию.
    \item и многое другое.
\end{itemize}

Часто позвонившего соединяли с оператором, который его консультировал по необходимому
вопросу. Но технический прогресс не стоит на месте и, чтобы уменьшить
затраты человеко-часов, освободить людей от рутинной работы и увеличить
количество обслуживаемых клиентов, были разработы технологии, которые позволяют
автоматизировать процессы, где требуется оператор для обслуживания клиентов.
Одно из таких решений~-~интерактивное голосовое меню~(\hyperlink{ivr}{IVR}),
которое с помощью предварительно записанных голосовых сообщений позволяло
позвонившему вести диалог с данной системой и, по возможности, решить его проблему
без участия человека~(оператора), либо же система перенаправляла на необходимого
оператора, который специализировался на проблеме позвонившего.

\textbf{IVR} (англ. Interactive Voice Response) — система предварительно записанных
голосовых сообщений, выполняющая функцию маршрутизации звонков внутри call-центра
с использованием информации, вводимой клиентом на клавиатуре телефона с помощью
тонального набора. Озвучивание IVR — важная составляющая успеха call-центра.
Правильно подобранное сочетание музыкального сопровождения, голоса диктора и
используемой лексики создаёт благоприятное впечатление от звонка в организацию.
Маршрутизация, выполняемая с помощью IVR-системы, обеспечивает правильную загрузку
операторов продуктов и услуг компании \cite{ivr}.

Система IVR позволяет улучшить качество обслуживание:
\begin{itemize}
    \item Балансировать звонки между свободными операторами.
    \item Получать дополнительную информацию.
    \item Уменьшить время ожидания.
    \item Обрабатывать звонки в нерабочее время.
\end{itemize}

Существует несколько способом взаимодействия с системой IVR:
\begin{itemize}
    \item Тональный набор (\hyperlink{dtmf}{DTMF})
    \item Распознавание речи
    \item Распознавание голоса
\end{itemize}

Тональный набор (DTMF) -- набор телефонного номера, при котором каждая цифра кодируется
парой из восьми тональных частот звукового диапазона. Во всех телефонах с
тональным набором обычно используется стандартная двенадцатикнопочная клавиатура.
Каждой строке в матрице клавиатуры соответствует низкочастотный тон, а каждому
столбцу - высокочастотный. При нажатии кнопки два тона суммируются и формируется
двухчастотный сигнал. Все частоты подобраны так, что ни одна из них не является
гармоникой другой. Длительность одного символа при тональном наборе~--~40~мс~\cite{dtmf}.

Распознавание речи — процесс преобразования речи (аналогового сигнала) в цифровую информацию.

Распознавание по голосу — одна из форм биометрической аутентификации, позволяющая
идентифицировать личность человека по совокупности уникальных характеристик голоса.
Относится к динамическим методам биометрии. Однако, поскольку голос человека может
меняться в зависимости от возраста, эмоционального состояния, здоровья,
гормонального фона и целого ряда других факторов, не является абсолютно точным~
\cite{recognize_voice}.

Так же одной из проблем текущего способа работы систем голосового меню~--~это
необходимость прослушать весь список возможных операций, а также трудность в
запоминании соотвествия пункта меню с действием.

С помощью голосового управления можно не озвучивать меню действий, а дать
пользователю возможность сообщить его цель звонка, после чего на основе
сказанного \textit{IVR} попробует проанализировать речь и маршрутизировать звонок.
