\chapter{ПОКАЗАТЕЛИ ПРОИЗВОДИТЕЛЬНОСТИ}

\section{Проведение экспериментов}
Для проведения экспериментов используется компьюетр со следующими характеристиками:
\begin{itemize}
    \item CPU Intel(R) Core(TM) i3-4160 CPU @ 3.60GHz;
    \item Оперативная память 8Гб;
    \item Операционная система ubuntu 18.04.
\end{itemize}

В рамках экспериметов не учитывается время инициализации, а только
процесс чтения данных и их распознавание.

\section{Результаты экспериментов}
На рисунках~\ref{fig:asr:time:short}--\ref{fig:asr:time:long} представлены
результаты экспериментов, на которых изображены графики зависимости времени
распознавания в зависимости от длительности аудио-файла.

\pic[H][0.5]{short.png}{asr:time:short}{Зависимость времени распознавания от длительности коротких фразы}
\pic[H][0.5]{long.png}{asr:time:long}{Зависимость времени распознавания от длительности длинных фразы}

\section{Оценка полученных результатов}
В результате проведённых экспериментов получены графики~\ref{fig:asr:time:short}--\ref{fig:asr:time:long},
которые показывают, что разработанная система использует умеренно количество
ресурсов, и допускается её использование совместно с другими приложениями,
однако запуск других ресурсоёмких приложений совместно с данными не рекомендуется,
по скольку потребление памяти \textit{ASR} могут резко возрасти при работе
с большим количество рабочих потоков.

Полученные результаты удовлетворяют установленным требованиям и позволяют
внедрить в эксплуатацию разработанную подсистему.
