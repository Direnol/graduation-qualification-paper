\chapter{ПОКАЗАТЕЛИ ПРОИЗВОДИТЕЛЬНОСТИ}

Далее на рисунках~\ref{fig:asr:time:short}--\ref{fig:asr:mem:long} приводятся
результаты экспериментов.

\section{Временные показатели}
На рисунках~\ref{fig:asr:time:short}--\ref{fig:asr:time:long} представлены
результаты экспериментов, на которых изображены графики зависимости времени
распознавания в зависимости от длительности аудио-файла.

\pic[h!][0.6]{short.png}{asr:time:short}{Зависимость времени распознавания от длительности коротких фразы}
\pic[h!][0.6]{short.png}{asr:time:long}{Зависимость времени распознавания от длительности длинных фразы}
\newpage
\section{Пространственные показатели}
На рисунках~\ref{fig:asr:mem:short}--\ref{fig:asr:mem:long} представлены
результаты экспериментов, на которых изображены графики зависимости потребления
памяти в зависимости от длительности аудио-файла.

\pic[h!][0.6]{short.png}{asr:mem:short}{Зависимость потребление памяти от длительности коротких фраз}
\pic[h!][0.6]{short.png}{asr:mem:long}{Зависимость потребления памяти от длительности длинных фразы}

\section{Оценка полученных результатов}
В результате проведённых экспериментов получены графики~\ref{fig:asr:time:short}--\ref{fig:asr:mem:long},
которые показывают, что разработанная система использует умеренно количество
ресурсов, и допускается её использование совместно с другими приложениями,
однако запуск других ресурсоёмких приложений совместно с данными не рекомендуется,
по скольку потребление памяти \textit{ASR} могут резко возрасти при работе
с большим количество рабочих потоков.

Каждый поток в среднем требует \dots MB для работы с короткими фразами и
\dots MB для длинных фраз.

Полученные результаты удовлетворяют установленным требованиям и позволяют
внедрить в эксплуатацию разработанную подсистему.
