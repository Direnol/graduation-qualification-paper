\chapter{ПРОЕКТИРОВАНИЕ СИСТЕМЫ}

\section{Используемые технологии}

\hyperlink{http}{HTTP} --- это прокол передачи данных. В настоящее время существует для передачи
произвольных данных и подразумевает наличие клиента и сервера~\cite{http}.
Клиенты инициализирует запрос, а серверы – ожидают соединения для передачи
запроса, производят какие-либо действия и посылают результат обратно. Прокол
используется во Всемирной паутине для получения информации с веб-сайтов. А
также протокол используется для для транспорта других протоколов с прикладным
уровнем. Основным объектом манипуляции является ресурс, на который указывается URI.

Nginx [engine x] —-- это HTTP-сервер и обратный прокси-сервер, почтовый
прокси-сервер, а также TCP/UDP прокси-сервер общего назначения, изначально
написанный Игорем Сысоевым~\cite{nginx}.

Прокси --- сервер (proxy-server) – это особый тип приложения, которое
выступает в роли посредника между клиентскими и серверными частями
распределенных сетевых приложений. Предполагается, что клиенты принадлежат
внутренней (защищаемой) сети, а серверы – внешней (потенциально опасной).

С++ --- язык общего назначения и задуман для того, чтобы настоящие программисты получили
удовольствие от самого процесса программирования. За исключением второстепенных деталей он
содержит язык С как подмножество. Язык С расширяется введением гибких и эффективных средств,
предназначенных для построения новых типов~\cite{cpp}

Kaldi --- это набор инструментов для распознавания речи, свободно доступный по
лицензии Apache. Kaldi стремится предоставить программное обеспечение, которое
является гибким и расширяемым~\cite{kaldi}.

libevent --- библиотека для асинхронного уведомления о событиях со всроенным
HTTP~--~сервером.

CMake — кроcсплатформенная утилита для автоматической сборки программы из
исходного кода. При этом сама CMake непосредственно сборкой не занимается,
а представляет из себя front-end. В качестве back-end`a могут выступать
различные версии make и Ninja~\cite{cmake}.

\subsection{Форматы аудио файлов}

\subsection{HTTP сервер}
Все взаимодействия в общей системе происходит по протоколу HTTP. Данный протокол
предполагает наличие сервера и клиента.

Клиентом в данном случае является любой пославший запрос на распознавнание.

\subsection{Распознавание речи}
\subsection{Nginx}
\section{Принцип работы системы}
\section{Способ взаимодействия с системой}
Как только системе IVR требуется получить ответ от собеседника, она
сообщает серверу маршрутизации аудио потоков(\hyperlink{msr}{MSR}), что необходимо
перенаправить последующий аудио-поток от позвонившего на сервер автоматического
распознования речи(\hyperlink{asr}{ASR}) по протоколу HTTP, после чего вернуть
ответ от ASR.

MSR пернапрвляет аудио-поток на ASR, добавляя заголовки, чтобы определить поведение
ASR:
\begin{itemize}
    \item sr-key -- Фразы, перечисленные через запятую, которые необходимо распознать
    \item Content-Type -- формат аудио-потока. Либо {\tt audio/x-wav} для wav формата,
    либо {\tt audio/x-pcm} для pcm формата
    \item Transfer-Encoding: chunked -- заголовок сообщающий, что будет потоковая передача
\end{itemize}
