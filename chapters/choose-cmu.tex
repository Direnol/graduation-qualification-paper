\section{CMU Sphinx}
\subsection{Сборка}

В листинге~\ref{cmu:build} представлены команды для скачивания и сборки \textit{CMU Sphinx}.
После данных действий необходимо скачать русскую языковую модель.

\begin{lstlisting}[caption={Клонирование и сборка необходимых интсрументов},label={cmu:build},language=bash]
$ git clone https://github.com/cmusphinx/sphinxbase.git
$ cd sphinxbase && ./autogen.sh && make && cd ..
$ git clone https://github.com/cmusphinx/pocketsphinx.git
$ cd pocketsphinx && ./autogen.sh && make && cd ..
$ git clone https://github.com/cmusphinx/sphinxtrain.git
$ cd sphinxtrain && ./autogen.sh && make && cd ..
\end{lstlisting}

В листинге~\ref{cmu:model} представлена иерархия языковой модели.

\begin{lstlisting}[caption={Иерархия языковой модели},label={cmu:model},language=bash]
$ tree zero_ru_cont_8k_v3
zero_ru_cont_8k_v3/
|- COPYING
|- decoder-test.sh
|- decoder-test.wav
|- README
|- ru.dic
|- ru.lm
|- zero_ru.cd_cont_4000
|   |- feat.params
|   |- feature_transform
|   |- mdef
|   |- means
|   |- mixture_weights
|   |- noisedict
|   |- transition_matrices
|   - variances
|- zero_ru.cd_ptm_4000
|   |- feat.params
|   |- mdef
|   |- means
|   |- mixture_weights
|   |- noisedict
|   |- sendump
|   |- transition_matrices
|   - variances
- zero_ru.cd_semi_4000
    |- feat.params
    |- mdef
    |- means
    |- mixture_weights
    |- noisedict
    |- sendump
    |- transition_matrices
    - variances
\end{lstlisting}

\subsection{Запуск}
Возможны два варианта запуска данного инструмента:
\begin{enumerate}
    \item Используется языкова модель, которая транслирует всю речь в текст;
    \item Используется грамматика, по которой происходит распознавание.
\end{enumerate}

Примеры запуска \textit{CMU Sphinx} в двух режимах представлено в
листингах~\ref{cmu:config:1} и~\ref{cmu:config:2} соответвенно.\\

\begin{lstlisting}[caption={Запуск \textit{CMU Sphinx}. Вариант 1},label={cmu:config:1}]
$ pocketsphinx_continuous \
    -hmm zero_ru_cont_8k_v3/zero_ru.cd_cont_4000 \
    -lm zero_ru_cont_8k_v3/ru.lm \
    -dict zero_ru_cont_8k_v3/ru.dic \
    -samprate 8000 \
    -remove_noise no \
    -nfft 512 -infile sample.wav
\end{lstlisting}

\begin{lstlisting}[caption={Запуск \textit{CMU Sphinx}. Вариант 2},label={cmu:config:2}]
$ pocketsphinx_continuous \
    -hmm zero_ru_cont_8k_v3/zero_ru.cd_cont_4000 \
    -jsgf ivr.jsgf \
    -dict zero_ru_cont_8k_v3/ru.dic \
    -samprate 8000 \
    -remove_noise no \
    -nfft 512 -infile yes_or_no.wav
\end{lstlisting}

Для запуска во втором режиме используется грамматика в формате \textit{jsgf}.
Инструкция \texttt{grammar} обозначает имя грамматики \texttt{ivr}. Далее
идёт перечисление возможных токенов: \texttt{<имя токена> = набор токенов или/и лексем}.
Символ \texttt{|} означает <<\textit{или}>>, то есть возможен любой токен из перечисленных.
Скобочки группируют лексемы и токены. Инструкция \texttt{public} обозначает целевой символ.
Рассмотрим пример представлен в листинге~\ref{cmu:jsgf}:

\begin{itemize}
    \item \texttt{grammar ivr}~---~обозначает имя грамматики;
    \item \texttt{<answer> = (да|нет)}~---~вводит новый токен \texttt{answer},
    который равен <<да>> или <<нет>>;
    \item \texttt{public <query> = (<answer> | <n1>)}~---~Означает, что целевой
        символ \texttt{query} может быть либо ответом: <<да>>,<<нет>>, либо цифрой.
\end{itemize}


\begin{minipage}{\linewidth}
\captionof{lstlisting}{Пример грамматики}\label{cmu:jsgf}
\begin{Verbatim}[fontsize=\footnotesize]
#JSGF V1.0;
grammar ivr;

<answer> = (да|нет);
<n1> = (ноль|один|два|три|четыре|пять|шесть|семь|восемь|девять);
public <query> = (<answer> | <n1>);

\end{Verbatim}
\end{minipage}

Как видно по результам работы~\ref{cmu:res:1} и~\ref{cmu:res:2} данный инструмент
справляется с распознаванием речи в обоих режимах.


\begin{lstlisting}[caption={Результат запуска 1 версии},label={cmu:res:1}, language=bash]
# родион потапыч отсчитывал каждый новый вершок углубления и давно определил про себя
\end{lstlisting}

\begin{lstlisting}[caption={Результат запуска 2 версии},label={cmu:res:2},language=bash]
# да да нет
\end{lstlisting}
\subsection{Конфигурация}

Настройка данного иснтрумента ограничена, по скольку языковая модель вводит свои
ограничения.
