\chapter{ВЫБОР ИНСТРУМЕНТА ДЛЯ РАСПОЗНАВАНИЯ РЕЧИ}

\section{Обзор готовых решений}

Имеется два готовых решения, которые удовлертворяют ранее представленным условиям:
\begin{enumerate}
    \item Kaldi
    \item CMU Sphinx
\end{enumerate}

Оба инструмента используют нейронные сети для преобразования аналогового сигнала
(речи) в цифровой вид и имеют готовые модели, что позволяет пропустить этап обучения
представленных инструментов на первоначальном этапе.

\section{Критерии для сравнение инструментов}
Для выбора инструмента требуется сравнить по следующим критериям:
\begin{itemize}
    \item Время распозвания речи.
    \begin{itemize}
        \item Короткие фразы.
        \item Длинные фразы.
    \end{itemize}
    \item Потребление ресурсов.
    \begin{itemize}
        \item Память.
        \item Нагрузка на процессор.
    \end{itemize}
    \item Точность распозвания.
    \begin{itemize}
        \item Для коротких фраз.
        \item Для длинных фраз.
    \end{itemize}
\end{itemize}
\section{Конфигурация инструментов}
\subsection{Kaldi}
\subsection{CMU Sphinx}
\section{Проведение экспериметов}
\section{Результаты экспериментов}
\section{Оценка полученных результатов}
