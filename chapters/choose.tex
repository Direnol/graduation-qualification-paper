\chapter{ВЫБОР ИНСТРУМЕНТА ДЛЯ РАСПОЗНАВАНИЯ РЕЧИ}

\section{Обзор готовых решений}

Имеется два готовых решения, которые удовлертворяют ранее представленным
критериям~(см. стр.~\pageref{enum:asr:cond}):
\begin{itemize}
    \item Kaldi
    \item CMU Sphinx
\end{itemize}

Оба инструмента используют нейронные сети для преобразования аналогового сигнала
(речи) в цифровой вид и имеют готовые модели для русского языка, что позволяет
пропустить этап обучения представленных инструментов на первоначальном этапе, и
сосредоточиться на реализации подсистемы голосовго управления.

\section{Критерии для сравнение инструментов}
Для выбора инструмента требуется сравнить их по следующим критериям и выбрать
наиболее подходящий:
\begin{itemize}
    \item Время распозвания речи.
    \begin{itemize}
        \item Короткие фразы.
        \item Длинные фразы.
    \end{itemize}
    \item Потребление ресурсов.
    \begin{itemize}
        \item Память.
        \item Нагрузка на процессор.
    \end{itemize}
    \item Точность распозвания.
    \begin{itemize}
        \item Для коротких фраз.
        \item Для длинных фраз.
    \end{itemize}
\end{itemize}

\section{Kaldi}
\subsection{Сборка}
Для сборки данного инструмента необходимо выполнить команды
представленные в листинге~\ref{kaldi:build}. Данный процесс займёт много времени.
Также необходимо скачать готовую языковую модель для русского языка, которая
будет использоваться для запуска \textit{kaldi}.

\begin{lstlisting}[caption={Сборка Kaldi}, label=kaldi:build]
$ git clone https://github.com/kaldi-asr/kaldi
$ cd kaldi
$ cd tools
$ extras/check_dependencies.sh
$ make
$ cd ../src
$ ./configure --shared
$ make
\end{lstlisting}

После сборки скачиваем языковую модель. Иерархия файлов модели представлена
в выводе команды \texttt{tree} в листинге~\ref{kaldi:model}. Данная модель
расположена в папке \texttt{/home/user/project/tdnn}.

\begin{lstlisting}[caption={Структура языковой модели}, label={kaldi:model}]
$ tree exp/tdnn
|- conf
|   |- ivector_extractor.conf
|   |- mfcc.conf
|   |- online_cmvn.conf
|   |- online.conf
|   └- splice.conf
|- final.mdl
|- frame_subsampling_factor
|- graph
|   |- disambig_tid.int
|   |- HCLG.fst
|   |- num_pdfs
|   |- phones.txt
|   └- words.txt
|- ivector_extractor
|   |- final.dubm
|   |- final.ie
|   |- final.mat
|   |- global_cmvn.stats
|   |- online_cmvn.conf
|   └- splice_opts
└- tree
\end{lstlisting}

\subsection{Запуск}
Для того, чтобы запустить данный инструмент для транслирования речи в текст~---~
воспользуемся скриптом для запуска~\ref{kaldi:run}.

\begin{lstlisting}[caption={Запуск Kaldi},label={kaldi:run},float=h]
#!/bin/bash
export KALDI_ROOT=${HOME}/project/kaldi

KALDI_PATH=$PWD/utils:$KALDI_ROOT/src/bin:$KALDI_ROOT/tools/openfst/bin
KALDI_PATH=${KALDI_PATH}:$KALDI_ROOT/src/fstbin:$KALDI_ROOT/src/gmmbin
KALDI_PATH=${KALDI_PATH}:$KALDI_ROOT/src/featbin:$KALDI_ROOT/src/lm
KALDI_PATH=${KALDI_PATH}:$KALDI_ROOT/src/sgmmbin:$KALDI_ROOT/src/sgmm2bin
KALDI_PATH=${KALDI_PATH}:$KALDI_ROOT/src/fgmmbin:$KALDI_ROOT/src/latbin
KALDI_PATH=${KALDI_PATH}:$KALDI_ROOT/src/nnetbin:$KALDI_ROOT/src/nnet2bin
KALDI_PATH=${KALDI_PATH}:$KALDI_ROOT/src/online2bin:$KALDI_ROOT/src/ivectorbin
KALDI_PATH=${KALDI_PATH}:$KALDI_ROOT/src/lmbin:$KALDI_ROOT/src/chainbin
KALDI_PATH=${KALDI_PATH}:$KALDI_ROOT/src/nnet3bin:$PWD:$PATH:$KALDI_ROOT/tools/sph2pipe_v2.5

export PATH=${PATH}:${KALDI_PATH}

export LC_ALL=C

online2-wav-nnet3-latgen-faster \
      --word-symbol-table=exp/tdnn/graph/words.txt \
      --frame-subsampling-factor=3 --frames-per-chunk=51 \
      --acoustic-scale=1.0 --beam=12.0 --lattice-beam=6.0 \
      --max-active=10000 --config=exp/tdnn/conf/online.conf \
      exp/tdnn/final.mdl exp/tdnn/graph/HCLG.fst \
      ark:decoder-test.utt2spk scp:decoder-test.scp ark:- |
    lattice-lmrescore --lm-scale=-1.0 ark:- 'fstproject \
    --project_output=true data/lang_test_rescore/G.fst |' ark:- |
    lattice-lmrescore-const-arpa ark:- data/lang_test_rescore/G.carpa ark:- |
    lattice-align-words data/lang_test_rescore/phones/word_boundary.int \
    exp/tdnn/final.mdl ark:- ark:- |
    lattice-to-ctm-conf --frame-shift=0.03 --acoustic-scale=0.08333 ark:- - |
    local/int2sym.pl -f 5 data/lang_test_rescore/words.txt - -
\end{lstlisting}

\subsection{Конфигурация}

\section{CMU Sphinx}
\subsection{Сборка}

В листинге~\ref{cmu:build} представлены команды для скачивания и сборки \textit{CMU Sphinx}.
После данных действий необходимо скачать русскую языковую модель.

\begin{lstlisting}[caption={Клонирование и сборка необходимых интсрументов},label={cmu:build},language=bash]
$ git clone https://github.com/cmusphinx/sphinxbase.git
$ cd sphinxbase && ./autogen.sh && make && cd ..
$ git clone https://github.com/cmusphinx/pocketsphinx.git
$ cd pocketsphinx && ./autogen.sh && make && cd ..
$ git clone https://github.com/cmusphinx/sphinxtrain.git
$ cd sphinxtrain && ./autogen.sh && make && cd ..
\end{lstlisting}

В листинге~\ref{cmu:model} представлена иерархия языковой модели.

\begin{lstlisting}[caption={Иерархия языковой модели},label={cmu:model},language=bash]
$ tree zero_ru_cont_8k_v3
zero_ru_cont_8k_v3/
|- COPYING
|- decoder-test.sh
|- decoder-test.wav
|- README
|- ru.dic
|- ru.lm
|- zero_ru.cd_cont_4000
|   |- feat.params
|   |- feature_transform
|   |- mdef
|   |- means
|   |- mixture_weights
|   |- noisedict
|   |- transition_matrices
|   - variances
|- zero_ru.cd_ptm_4000
|   |- feat.params
|   |- mdef
|   |- means
|   |- mixture_weights
|   |- noisedict
|   |- sendump
|   |- transition_matrices
|   - variances
- zero_ru.cd_semi_4000
    |- feat.params
    |- mdef
    |- means
    |- mixture_weights
    |- noisedict
    |- sendump
    |- transition_matrices
    - variances
\end{lstlisting}

\subsection{Запуск}
Возможны два варианта запуска данного инструмента:
\begin{enumerate}
    \item Используется языкова модель, которая транслирует всю речь в текст;
    \item Используется грамматика, по которой происходит распознавание.
\end{enumerate}

Примеры запуска \textit{CMU Sphinx} в двух режимах представлено в
листингах~\ref{cmu:config:1} и~\ref{cmu:config:2} соответвенно.

\begin{lstlisting}[caption={Запуск \textit{CMU Sphinx}. Вариант 1},label={cmu:config:1}]
$ pocketsphinx_continuous \
    -hmm zero_ru_cont_8k_v3/zero_ru.cd_cont_4000 \
    -lm zero_ru_cont_8k_v3/ru.lm \
    -dict zero_ru_cont_8k_v3/ru.dic \
    -samprate 8000 \
    -remove_noise no \
    -nfft 512 -infile sample.wav
\end{lstlisting}

\begin{lstlisting}[caption={Запуск \textit{CMU Sphinx}. Вариант 2},label={cmu:config:2}]
$ pocketsphinx_continuous \
    -hmm zero_ru_cont_8k_v3/zero_ru.cd_cont_4000 \
    -jsgf ivr.jsgf \
    -dict zero_ru_cont_8k_v3/ru.dic \
    -samprate 8000 \
    -remove_noise no \
    -nfft 512 -infile yes_or_no.wav
\end{lstlisting}

Для запуска во втором режиме используется грамматика в формате \textit{jsgf}.
Инструкция \texttt{grammar} обозначает имя грамматики \texttt{ivr}. Далее
идёт перечисление возможных токенов: \texttt{<имя токена> = набор токенов или/и лексем}.
Символ \texttt{|} означает <<\textit{или}>>, то есть возможен любой токен из перечисленных.
Скобочки группируют лексемы и токены. Инструкция \texttt{public} обозначает целевой символ.
Рассмотрим пример представлен в листинге~\ref{cmu:jsgf}:

\begin{itemize}
    \item \texttt{grammar ivr}~---~обозначает имя грамматики;
    \item \texttt{<answer> = (да|нет)}~---~вводит новый токен \texttt{answer},
    который равен <<да>> или <<нет>>;
    \item \texttt{public <query> = (<answer> | <n1>)}~---~Означает, что целевой
        символ \texttt{query} может быть либо ответом: <<да>>,<<нет>>, либо цифрой.
\end{itemize}

{
\captionof{lstlisting}{Пример грамматики}\label{cmu:jsgf}
\vspace{-10pt}
\begin{Verbatim}[fontsize=\footnotesize,xleftmargin=15mm]
#JSGF V1.0;
grammar ivr;
<answer> = (да|нет);
<n1> = (ноль|один|два|три|четыре|пять|шесть|семь|восемь|девять);
public <query> = (<answer> | <n1>);
\end{Verbatim}
}

Как видно по результам работы~\ref{cmu:res:1} и~\ref{cmu:res:2} данный инструмент
справляется с распознаванием речи в обоих режимах.


\begin{lstlisting}[caption={Результат запуска 1 версии},label={cmu:res:1}, language=bash]
# родион потапыч отсчитывал каждый новый вершок углубления и давно определил про себя
\end{lstlisting}

\begin{lstlisting}[caption={Результат запуска 2 версии},label={cmu:res:2},language=bash]
# да да нет
\end{lstlisting}
\subsection{Конфигурация}

Настройка данного иснтрумента ограничена, по скольку языковая модель вводит свои
ограничения. Тем не менее, остаются параметры для установки частоты дискретизации
и подавления шума:
\begin{enumerate}
    \item \texttt{samprate}~---~частота дискретизации;
    \item \texttt{remove\_noise}~---~подавление шума.
\end{enumerate}

