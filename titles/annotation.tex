\begin{center}
    \bf
    АННОТАЦИЯ
\end{center}


\noindent
Выпускная квалификационная работа \fioa \linebreak
по теме <<\topicname>>.\linebreak
\hfill \\
Объём работы \pageref{LastPage} страниц, на которых размещены \dots рисунков
и 0 таблиц. При написании работы использовалось \total{citenum} источника. \\
\hfill \\
Ключевые слова: ASR, Kaldi, CMU Sphinx \\
\hfill \\
Работа выполнена на кафедре ВС СибГУТИ и в лаборатории IMS предприятия Элтекс. \\
Руководитель~--~\theadpos~\thead, \\


Целью бакалаврской работы была разработка подсистемы голосовго управления
для маршрутизацией звонков для абоненсткого шлюза ECSS10.


Для реализации поставленной задачи требовалось:
\begin{itemize}
    \item Изучить инструменты для распозвания речи.
    \item Освоить системы сборки проектов.
    \item Спроектировать архитектуру подсистемы.
    \item Описать внешний интерфейс.
    \item Разработать спроектированную систему.
\end{itemize}


В рамках выпускной квалификационной работы были изучены инструменты для
распознавания речи, и их интеграция в подсистему голосового управления
маршрутизацией звонков для абоненсткого шлюза ECSS10.


\thispagestyle{empty}
