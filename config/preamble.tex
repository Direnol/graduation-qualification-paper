\documentclass[a4paper]{report}
\PassOptionsToPackage{russian,english}{babel}
\usepackage[14pt]{extsizes} % для того чтобы задать нестандартный 14-ый размер шрифта
\usepackage[utf8]{inputenc}
\usepackage[T2A]{fontenc}
\usepackage{fontspec} % XeTeX
\usepackage{xunicode} % Unicode для XeTeX
\usepackage{xltxtra}  % Верхние и нижние индексы
\usepackage{pdfpages} % Вставка PDF
% Шрифты, xelatex
\defaultfontfeatures{Ligatures=TeX}
\setmainfont{Times New Roman}
\newfontfamily\cyrillicfont{Times New Roman}
%\setsansfont{Liberation Sans}
\setmonofont{FreeMono} % Моноширинный шрифт для оформления кода
% Русский язык
\usepackage{polyglossia}
\setdefaultlanguage{russian}
\usepackage[english,russian]{babel}
\usepackage{setspace,amsmath}
\usepackage[left=2cm, top=15mm, right=15mm, bottom=15mm, nohead, footskip=10mm]{geometry} % настройки полей документа
\usepackage{indentfirst} % отделять первую строку раздела абзацным отступом
\usepackage{textcomp}
\usepackage{titlesec}
\usepackage[pdfborder={0 0 0}, colorlinks=false]{hyperref}
\usepackage{listings}
\usepackage{graphicx}
\usepackage[justification=centering]{caption}
\usepackage{color}
\usepackage{xcolor}
\usepackage{ifthen}
\usepackage{xstring}
\usepackage{enumitem}
\usepackage{amsmath}
\usepackage{xfrac}
\usepackage{csvsimple}
\usepackage{tikz}
\usepackage{url}
\usepackage{cite}
\usepackage{multirow}
\usepackage{lastpage}
\usepackage{lipsum}
\usepackage{titlesec}
\usepackage{fancyvrb}
\usepackage{longtable}
\usepackage{tabularx}
\bibliographystyle{ugost2008}
\patchcmd{\thebibliography}{\chapter*}{\section*}{}{}

\makeatletter
\renewcommand{\@biblabel}[1]{#1.}
\makeatother


\addto\captionsrussian{\renewcommand{\figurename}{Рисунок}}
\DeclareCaptionLabelSeparator{dash}{~---~}
\captionsetup{
   labelsep=dash
}
\usepackage{totcount}
\newtotcounter{citenum}
\newtotcounter{picnum}
\def\oldcite{}
\let\oldcite=\bibcite
\def\bibcite{\stepcounter{citenum}\oldcite}

% Настройка вертикальных и горизонтальных отступов
\titlespacing{\chapter}{\parindent}{-30pt}{12pt}[45pt]
\titlespacing{\section}{\parindent}{6pt}{6pt}[45pt]
\titlespacing{\subsection}{\parindent}{12pt}{3pt}[45pt]

\usepackage{secdot}

% Настройка заголовков
\titleformat{\chapter}
{\Large\hyphenpenalty=10000}
{\thechapter~}{0pt}{\Large}

\titleformat*{\chapter}
{\Large}

\titleformat{\section}
{\large}
{\thesection~}
{0pt}{}

\titleformat*{\section}
{\large}

\titleformat{\subsection}
{\normalsize}
{\thesubsection~}
{0pt}{}

\titleformat*{\subsection}
{\normalsize}

\usepackage{tocloft}

\renewcommand{\cfttoctitlefont}{\hspace{0.4\textwidth}} %уменьшаем размер шрифта
% и ровняем по центру межстрочные отступы в Оглавлении:

\setlength{\cftbeforetoctitleskip}{-5mm} %отступ Оглавления от верхнего поля страницы.
\setlength{\cftbeforechapskip}{1mm}

% Отступы от левого поля:
\setlength{\cftchapindent}{0mm} %отступ между левым полем и \chapter{}
\setlength{\cftsecindent}{0mm} %отступ между левым полем и \section{title}
\setlength{\cftsubsecindent}{0mm} %отступ между левым полем и \section{title}
\setlength{\cftsecnumwidth}{1.5cm}
\setlength{\cftsubsecnumwidth}{3cm}
% Отточия в Оглавлении
\renewcommand\cftchapdotsep{\cftdot} %добавляет отточия после \chapter{title}
\renewcommand{\cftchapleader}{\cftdotfill{\cftchapdotsep}} %делает отточия после \chapter{title} тонкими, (по умолчанию жирные).
\renewcommand\cftsecdotsep{\cftdot} %делает отточия после \section{title} частыми.
\renewcommand\cftsubsecdotsep{\cftdot} %делает отточия после \section{title} частыми.

\renewcommand{\cftchapfont}{\normalfont} % убирает \bf в оглавлении у chapter
\usepackage[tocflat]{tocstyle}
\usepackage{setspace}
\singlespacing
\setlist{nolistsep,itemsep=0cm,parsep=0pt,leftmargin=1.1cm}
\setlist[enumerate]{nolistsep,itemsep=0cm,parsep=0pt,leftmargin=1.3cm}

\let\OLDthebibliography\thebibliography
\renewcommand\thebibliography[1]{
    \OLDthebibliography{#1}
    \setlength{\parskip}{0pt}
    \setlength{\itemsep}{0pt plus 0.3ex}
}
\fvset{xleftmargin=\parindent}

% % ВИСЯЧИЕ СТРОКИ
\clubpenalty=10000 %управляет висячими строками в начале абзаца, по умолчанию 150. Чем больше число, тем меньше вероятность появления одиночных строк в конце листа, они если это возможно будут перенесены на следующий лист. Максимальное число 10000 - полный запрет висячих строк.
\widowpenalty=10000 %аналогично, но в отношении последних строк абзаца.

% \makeatletter
% \renewcommand{\l@section}{\@dottedtocline{1}{0.4cm}{0.4cm}}
% \renewcommand{\thesection}{\arabic{section}}
% \renewcommand{\section}{\@startsection{section}{1}{1.25cm}{-3.5ex plus -1ex minus -.2ex}{2.3ex plus.2ex}{\raggedright\normalfont\bfseries}}
% \makeatother

% \makeatletter
% \renewcommand{\l@subsection}{\@dottedtocline{2}{0.8cm}{0.8cm}}
% \renewcommand{\thesubsection}{\arabic{section}.\arabic{subsection}}
% \renewcommand{\subsection}{\@startsection{subsection}{2}{1.25cm}{-3.5ex plus -1ex minus -.2ex}{2.3ex plus.2ex}{\raggedright\normalfont\bfseries}}
% \makeatother

% Оформление листинга
\renewcommand{\lstlistingname}{Листинг}
\lstset{stepnumber=2,
    basicstyle=\footnotesize\ttfamily,
    captionpos=t,
    % frame=single,
    % numbers=left,
    inputencoding=utf8,
    extendedchars=\true,
    % belowcaptionskip=\parindent,
    aboveskip=6pt,
    xleftmargin=\parindent,
    showstringspaces=false,
    keywordstyle=,
    commentstyle=,
    identifierstyle=,
    stringstyle=,
    breaklines=true,
    texcl=true,
    float=h!
}
\DeclareCaptionFormat{fmt}{\parbox{\linewidth}{\hspace{8mm}#1#2#3}}
\captionsetup[lstlisting]{format=fmt,justification=raggedright}

\makeatletter
\def\verbatim{\footnotesize\@verbatim \frenchspacing\@vobeyspaces \@xverbatim}
\makeatother
\usepackage{fancyhdr}
\fancyhf{}
\pagestyle{fancy}
\fancyfoot[R]{\thepage} %prints the page number on the right side of the header
\fancypagestyle{plain}{}
\renewcommand{\headrulewidth}{0pt} % убрать разделительную линию

%%%%%%%%%%%%%%%%%%%%%%%%%%%%%%%%%%%%%%%%%%%%%%%%%%%%%%%%%%%%%%%%%%%%%%%%%%%%%%%%
\def \topicname {Разработка \mbox{системы} голосового управления маршрутизацией звонков
                для \mbox{абонентского} шлюза ECSS10}

\def \fio {Мингазеев~Р.Л.}
\def \fioa {Мингазеева~Р.Л.}
\def \fiou {Мингазееву~Р.Л.}

\def \faculty {ИВТ}
\def \group {ИВ-521}

\def \theadpos {ст. преподователь}
\def \thead {Крамаренко~К.Е.}
\def \chief {Курносов~М.Г.}
\def \chiefpos {д.\,т.\,н.~доцент}

\def \town {Новосибирск}
\def \year {2019}
%%%%%%%%%%%%%%%%%%%%%%%%%%%%%%%%%%%%%%%%%%%%%%%%%%%%%%%%%%%%%%%%%%%%%%%%%%%%%%%%
\newcommand{\chap}[1]{
    \chapter*{\parbox{0.93\linewidth}{\centering #1}}
    \addcontentsline{toc}{chapter}{#1}
}
\newcommand{\csection}[1]{
    \section*{\parbox{0.93\linewidth}{\centering #1}}
}
\NewDocumentCommand{\pic}{O{h} O{1} mmm}{\def\lableName{fig:#4}
    \stepcounter{picnum}
    \begin{figure}[#1]
        \centering
        \includegraphics[scale=#2]{#3}
        \caption{#5}\label{\lableName}
    \end{figure}
}

\fancypagestyle{firststyle}
{
    \fancyfoot[C]{
        \normalsize
        \town\,--\,\year
    }
    \fancy
}
